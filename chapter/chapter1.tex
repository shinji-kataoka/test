\section{背景}
システム同定とは,入出力のデータからシステムのダイナミクスを推定することであり,制御システムの効率的かつ高精度な設計,システムの動作予測,故障診断などにおいて重要な役割を果たす\cite{SSSS, SITF}.
特に線形システムは,その扱いやすさから製造業におけるロボットの制御や自動車産業における安定性制御,経済学では市場予測や価格変動モデルなど幅広い分野で近似モデルとして採用されている.
これまでシステム同定の研究では,収集された入出力データのサンプル数を無限に仮定した漸近的な解析が主流であり,入出力のデータ数が無限に増加する場合における性能保証が中心的なテーマだった.
しかし,現実においては利用可能なデータが有限である場合が一般的であり,漸近的な結果に基づく設計では実用上の精度が保証されないことがある.また,故障確率や過渡現象などの概念を解析するのに適しているといった理由で,非漸近的な解析が注目されるようになり,有限のデータにおける特性を解析する研究が行われてきた.

近年の研究動向として,高次元統計\cite{HDPA, HDSA}や機械学習理論\cite{TEOS}の発展に伴い,これらのツールを活用することで非漸近的な解析における理論的基盤がさらに強化された.特に,線形システムにおける有限の入出力データを用いた解析に関する理論が整備され,システムパラメータを所定の精度レベルで同定するために必要なデータ数の下限を導出するための情報理論的議論もされている.これにより,システムの学習効率を定性的に評価できるようになった\cite{SLTF}.

その中でも,線形時不変システムの同定は,構造がシンプルでありながら広範な応用が可能であるため,従来から盛んに研究されてきた.
特に入出力データから得られるシステム行列の最小二乗推定値について,誤差境界やサンプル複雑性が議論されてきた.
線形システムにおいて最も基本的なケースである,入力を印加しない完全観測システムについて,システム行列の最小二乗推定値に関する有限時間での推定誤差の上界や任意の精度を保証するために必要なデータ数の下界を導出している\cite{FTIIU, FTIOL}.
また,入力を印加するシステムについてシステム行列の誤差を制御設計に適用し最適二次レギュレータを用いて安定化制御を可能にする推定誤差とサンプル複雑性について議論されている\cite{OTSCO}.

入出力データからシステム同定をする手法の一つとして,パルス伝達関数の係数であるマルコフパラメータを推定する手法がある.
システムが安定な場合についてひとつの軌道の入出力データから,安定性を要求しない場合について複数の独立した軌道の入出力データから,それぞれマルコフパラメータ行列の最小二乗推定値の推定誤差の上界を導出している\cite{RHKB, NAIOL}.


\section{目的}
システムの入出力関係を示すマルコフパラメータを推定することは,直接的にシステムの入出力の関係を表すことができる点から,重要である.従来の手法では,入力が正規分布に従って生成されるというケースが一般的であり,入力信号に制約を設けず,理想的に設計可能であることを前提としている.
一方,現実の物理的なシステムでは,理想的な乱数に従う入力信号を生成することが困難な場合や,アクチュエータの物理的制約によって入力の大きさに上限が課される場合,電池駆動のシステムにおける電流の制限など,実際の応用環境では様々な制約が課されることが多い.そのため入力に制約がある場合でも適用可能な理論的保証が求められる.
このため本研究では,物理的な制約を考慮し,入力の大きさに上界がある条件下において有限の入出力データから高精度な推定値を得る入力設計手法を確立し,理論的な保証を与えることを目的とする.

\section{構成}
本論文の構成は以下の通りである.
まず,第2章では,マルコフパラメータおよび対象とするシステムの設定など,本論文で用いる数学的概念の定義を行い,マルコフパラメータの最小二乗推定の手順について述べる. 
また,先行研究の入力が正規分布に従い生成される場合のマルコフパラメータの推定精度について紹介する.
第3章では入力の大きさに上限がある場合にも適用可能な,本論文で提案する入力設計アルゴリズムを紹介し,数値実験から最良の入力を得ることのできる確率分布について検討している.
また,入力が決定した場合のマルコフパラメータ行列の推定誤差のスペクトルノルムについての上界を導出している.
最後に第4章で結論を述べる.

