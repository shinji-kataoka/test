\section{問題設定}
以下の離散時間線形時不変システム
\begin{align}
    \label{state1}
    x_{t+1} &= Ax_t + Bu_t + w_t \\
    \label{output1}
    y_t &= Cx_t + Du_t + z_t
\end{align}
を考える.ただし,
$x_t \in \mathbb{R}^n,u_t \in \mathbb{R}^p,y_t \in \mathbb{R}^m$はそれぞれ時刻$t$におけるシステムの状態,入力,出力を表す.また,$w_t \in \mathbb{R}^n,z_t \in \mathbb{R}^m$はプロセスノイズおよび観測ノイズを表し,$w_t \sim \mathcal{N}(0, \sigma_w^2I), z_t \sim \mathcal{N}(0, \sigma_z^2I)$である.
初期状態$x_1 = 0$であり,$\rho(A) \leq \eta <1$を仮定する.ここで,$\rho(\cdot)$は行列の最大固有値の絶対値を表す.

本章では,未知の行列$A,B,C,D$を推定することを目的とする.

\section{マルコフパラメータについて}
システム\eqref{state1}, \eqref{output1}において,入出力関係は
\begin{equation*}
    y_t = Du_t+ \sum_{i = 1}^{t-1}CA^{i-1}Bu_{t-i} + \sum_{i = 1}^{t-1}CA^{i-1}w_{t-i}
\end{equation*}
と表される.このときの入力の係数行列である$D, CB, CAB, CA^2B, \ldots$をシステムのマルコフパラメータと呼ぶ.



また,システムの最初の$T$個のマルコフパラメータを並べた行列を
\begin{equation}
    G = 
    \begin{bmatrix}
        D & CB & CAB & \ldots & CA^{T-2}B
    \end{bmatrix}
    \in \mathbb{R}^{m \times pT}
\end{equation}
と定義し,この$G$の推定を目的とする.
次の節では,この$G$の最小二乗推定の手順について説明する.

\section{マルコフパラメータの最小二乗推定の手順}

まず,推定手順を説明するために,データ収集の方法について説明する.
今回,システムの入出力データを時間長$\bar{N}$まで取得する。このとき得られるデータセットは  
\begin{equation*}
    \{(y_t, u_t):t = 1, 2, \ldots,  \bar{N} \}
\end{equation*}
である.
この与えられたデータセットから長さ$T$のセットを$N$個生成する.
$\bar{N} = N + T - 1$である.
ある時刻$t$において入力,ノイズを時間長$T$分並べたものを
\begin{align}
    \bar{u}_t &= 
    \begin{bmatrix}
        u_t^* & u_{t-1}^* & \ldots & u_{t-T+1}^*
    \end{bmatrix}^*
    \in \mathbb{R}^{pT}
    \\
    \bar{w}_t &= 
    \begin{bmatrix}
        w_t^* & w_{t-1}^* & \ldots & w_{t-T+1}^*
    \end{bmatrix}^*
    \in \mathbb{R}^{nT}
\end{align}
と表記する.
$y_t$を$t-T+1$まで再帰的に展開すると,出力$y_t$と$\bar{u}_t$,$G$の関係は
\begin{align}
    y_t &= Cx_t + Du_t + z_t \notag \\
        &= C(Ax_{t-1} + Bu_{t-1} + w_{t-1}) + Du_t + z_t \notag  \\
        \label{u_y}
        &= G\bar{u}_t + F\bar{w}_t + z_t + e_t 
\end{align}
となる.ここで,
$F = 
\begin{bmatrix}
    0 & C & CA & \ldots & CA^{T-2}
\end{bmatrix}\in \mathbb{R}^{m \times nT}$
であり,
$e_t = CA^{T-1}x_{t-T+1}$
は,時刻$t-T+1$におけるシステムの状態の影響による誤差に相当する.
この関係を用いて,$N$個の$(\bar{u}_t, y_t)_{t = T}^{\bar{N}}$を入出力データとして扱う.

上の\eqref{u_y}式右辺の2項目以降をノイズとして最小二乗法を適用すると$G$の最小二乗推定値は,
\begin{equation}
\label{least_square}
    \hat{G} = \underset{X \in \mathbb{R}^{m \times pT}}{\text{arg min}} \sum_{t = T}^{\bar{N}} \| y_t - X\bar{u}_t \|^2 
\end{equation}
と表せる.
ただし,$\| \cdot \|$はスペクトルノルムを表す.
また,
\begin{align}
    Y &=
    \begin{bmatrix}
        y_T & y_{T+1} & \ldots & y_{\bar{N}}
    \end{bmatrix}^*\in \mathbb{R}^{N \times m}  \\
    \label{U}
    U &=
    \begin{bmatrix}
        \bar{u}_T & \bar{u}_{T+1} & \ldots & \bar{u}_{\bar{N}}
    \end{bmatrix}^*\in \mathbb{R}^{N \times pT}
\end{align}
と定義すると,\eqref{least_square}式より,
\begin{equation}
    \hat{G} = \underset{X \in \mathbb{R}^{m \times pT}}{\text{min}}  \| Y - UX^* \|_F^2
\end{equation}
となる.なお,$\|\cdot\|_F$はフロベニウスノルムを表す.
最小二乗解である$\hat{G}$は,
\begin{equation}
    \label{hatG}
    \hat{G} = (U^{\dag}Y)^*
\end{equation}
と与えられる.
ここで,$U^{\dag} = (U^*U)^{-1}U^*$は左疑似逆行列である.また,
\begin{align}
    \label{W_norm}
    W &=
    \begin{bmatrix}
        \bar{w}_T & \bar{w}_{T+1} & \ldots & \bar{w}_{\bar{N}}
    \end{bmatrix}^*\in \mathbb{R}^{N \times pT}\\
    \label{E}
    E &=
    \begin{bmatrix}
        e_T & e_{T+1} & \ldots & e_{\bar{N}}
    \end{bmatrix}^*\in \mathbb{R}^{N \times m}\\
    Z &=
    \begin{bmatrix}
        z_T & z_{T+1} & \ldots & z_{\bar{N}}
    \end{bmatrix}^*\in \mathbb{R}^{N \times m}
\end{align}
と定義すると,システムの入出力関係を以下のように記述することができる.
\begin{equation}
    Y = UG^*+E + Z + WF^*
\end{equation}
$G$について解くと$G^* = U^{\dag}(Y - E - Z - WF^*) $となるため,
(\ref{hatG})式より最小二乗推定値と真値の誤差は,
\begin{equation}
    \label{error}
    (\hat{G} - G)^* = (U^*U)^{-1}U^*(E + Z +WF^*) 
\end{equation}
となる.

\section{先行研究における推定精度}
先行研究では入力を$\mathcal{N}(0, \sigma_u^2I_p)$に従い生成した場合について$G$の最小二乗推定
を行っていた.
この設定における最小二乗推定値$\hat{G}$と$G$の推定誤差のスペクトルノルムの上界を与えた定理を紹介する.
\begin{thm}
% \cite[Theorem3.2]{RHKB}
\cite{RHKB}
\label{thm_prev_error_bound}
    $c, C >0$は十分大きな定数である.
    $\rho(A) < 1$であり,$\frac{N}{\log^2{(Np)}} \geq cpT\log^2{(pT)}$であるとする.
    $q = p+n$,$N_w = cTq\log^2{Tq}\log^2{Nq}$とする.このとき,高い確率でマルコフパラメータ行列$G$の最小二乗推定値$\hat{G}$について,
    \begin{equation*}
        \| \hat{G}-G \| \leq \frac{R_w\|F\| + R_z + R_e}{\sigma_u\sqrt{N}}
    \end{equation*}
    が成り立つ.ここで,
    \begin{align*}
        R_z &= 8\sigma_z \sqrt{Tp + m} \\
        R_w &= \sigma_w (\sqrt{N_w} + N_w/\sqrt{N}) \\
        R_e &= C\sigma_e \sqrt{ \left( 1+\frac{mT}{N(1-\rho(A)^{T/2})}  \right)(Tp + m) }
    \end{align*}
    である.また,
    \begin{align*}
        \Phi(A) &= \sup_{\tau \geq 0} \frac{\|A^{\tau}\|}{\rho(A)^{\tau/2}} \\
        \Gamma_{\infty} &= \sum_{i = 0}^{\infty} \sigma_w^2A^i(A^*)^i + \sigma_uA^iBB^*(A^*)^i
    \end{align*}
    とすると,
    \begin{equation*}
        \sigma_e = \Phi(A)\|CA^{T-1}\|\sqrt{\frac{T\|\Gamma_{\infty}\|}{1-\rho(A)^T}}
    \end{equation*}
    である.
\end{thm}

この定理では,十分に大きな$N$において,サンプル数$N$に対して$\mathcal{O}(N^{-1/2})$で推定誤差の上界が減少することが示されている.
さらに,最小二乗推定した$\hat{G}$を用いて出力の予測誤差を示している.

\begin{sub}
% \cite[lemma3.3]{RHKB}
\cite{RHKB}
\label{sub_estimate_y}
    $G$の最小二乗推定値$\hat{G}$を用いて,観測出力$y_t$の推定値を
    \begin{equation*}
        \hat{y}_t = \hat{G}\bar{u}_t 
    \end{equation*}
    とする.このとき,
    \begin{equation*}
        \mathbb{E}[\|y_t-\hat{y}_t\|^2] \leq \sigma_w^2\|F\|_F^2 + \sigma_u^2\|G-\hat{G}\|_F^2 + m\sigma_z^2 + \|CA^{T-1}\|^2 \text{tr}(\Gamma_{\infty})
    \end{equation*}
    である.
\end{sub}


また,先行研究\cite{RHKB}において提案されている,行列$G$から元のシステムの行列$A, B, C, D$を実現するアルゴリズムを紹介する.
\begin{algorithm}
\caption{Ho-Kalman アルゴリズムによる状態空間実現}
\label{alg:ho-kalman}
\begin{algorithmic}[1]
\Procedure{Ho-Kalman の最小実現}{}
    \State \textbf{Inputs:} 長さ $T$,マルコフパラメータの最小二乗推定値 $\hat{G}$,システムの次数
    % \hspace{2.5cm}  $n$,ハンケル行列のサイズ $(T_1, T_2 + 1)$,ただし $T_1 + T_2 + 1 = T$
    \Statex \hspace{2.0cm}$n$,ハンケル行列のサイズ $(T_1, T_2 + 1)$,ただし $T_1 + T_2 + 1 = T$
    \State \textbf{Outputs:} 状態空間実現 $\hat{A}, \hat{B}, \hat{C}$
    \State ハンケル行列 $\hat{H} \in \mathbb{R}^{mT_1 \times pT_2}$ を $\hat{G}$ から生成
    \State $\hat{H}_1 \gets \hat{H}$ の最初の $pT_2$ 列
    \State $\hat{L} \gets \hat{H}_1$ のランク $n$ の近似行列
    \State $U, \Sigma, V \gets \text{SVD}(\hat{L})$
    \State $\hat{O} \gets U_{:, 1:n} \Sigma^{1/2}$
    \State $\hat{Q} \gets \Sigma^{1/2} V_{:, 1:n}^*$
    \State $\hat{C} \gets \hat{O}$ の最初の $m$ 行
    \State $\hat{B} \gets \hat{Q}$ の最初の $p$ 列
    \State $\hat{H}^{+} \gets \hat{H}$ の最後の $pT_2$ 列
    \State $\hat{A} \gets \hat{O}^{\dagger} \hat{H}^{+} \hat{Q}^{\dagger}$
    \State \Return $\hat{A} \in \mathbb{R}^{n \times n}, \hat{B} \in \mathbb{R}^{n \times p}, \hat{C} \in \mathbb{R}^{m \times n}$
\EndProcedure
\end{algorithmic}
\end{algorithm}

ブロック行列 $X = [X_1, X_2, \dots, X_T] \in \mathbb{R}^{m \times Tp}$ と,整数 $T_1, T_2$ (ただし $T_1 + T_2 \leq T$)が与えられたとき,
ここで用いるハンケル行列は,$m \times p$ サイズのブロックを持つ $T_1 \times T_2$ のブロック行列として定義され,$(i, j)$ 番目のブロックは,
\begin{equation*}
    H[i, j] = X_{i+j}
\end{equation*}
である.

また,ここから得られる行列の組$(\hat{A}, \hat{B}, \hat{C}, \hat{D})$の推定精度は,アルゴリズムの入力である$\hat{G}$の推定精度に依存しているため$G$をより厳密に推定することが目標となる.\cite{RHKB}
