
線形時不変システム同定の有限時間での解析においては,様々な観点から研究が行われている.本研究では,特に部分観測の線形時不変システムにおけるマルコフパラメータの推定に焦点を当てた.従来の研究では,入力信号が正規分布に従い生成されるという理想的な前提のもとで設計されていた.しかし,実際の応用環境では,アクチュエータの出力制約などにより,入力信号に様々な制約が課されることが一般的である.

そこで本研究では,より現実的な問題設定として,アクチュエータの物理的制約により入力信号の大きさに上限がある状況下で,システムの入出力関係を示すマルコフパラメータを推定する手法を提案し,その有効性を数値実験により検証した.具体的には,補題\ref{sub_input_alg}のアルゴリズムを用いる際に,異なる確率分布を用いて入力を設計し,推定誤差を比較した.その結果,一様分布や正規分布と比較して,分散の大きい丸めた正規分布や$\{−1, +1\}$を等確率でとるベルヌーイ試行など,上限値に高確率で近づく確率分布を用いることで,推定精度が向上することが明らかになった.また,推定するマルコフパラメータの数や用いる入出力データセットの数が少ない場合においても,問題1を解いた結果から,ベルヌーイ試行を用いた入力設計が有効であることが確認された.さらに,設計した入力を用いてマルコフパラメータの最小二乗推定を行った際の推定誤差のノルムについて,その理論的な誤差上界を導出した.これにより,制約のある環境下でも適切な入力設計を行うことで推定精度が保証されることを数理的に示した.

今後の課題として,同じ問題設定のもとでより厳密な推定誤差の上界を導出することが挙げられる.また,本研究ではアクチュエータの物理的制約に基づいた入力設計を行ったが,今後は電力消費が制約されるデバイスにおけるエネルギー制約を考慮した入力設計へと発展させることが求められる.さらに,実際の計測環境ではノイズが理想的な正規分布に従わない場合があるため,より現実的なノイズモデルを考慮した解析も重要な課題である.